\documentclass{article}

\usepackage[scientific-notation=true, binary-units=true]{siunitx}
\usepackage{amsmath}
\usepackage{pgfplots}
\sisetup{per-mode=fraction}
\sisetup{scientific-notation=false}
\pgfplotsset{compat=1.13}

\title{SYSC 4602 Assignment 3}
\date{November 25th, 2016}
\author{Jessica Morris \(100882290\)}

\begin{document}

\maketitle

\begin{enumerate}

\item For pure ALOHA, maximum throughput occurs at $ G = 0.5 $, with $ S = \frac{1}{2}e $. This means a channel utilization of 0.184. This gives a usable bandwidth of $ 0.184 \times \SI{56}{\kilo\bit\per\second} = \SI{10.3}{\kilo\bit\per\second} $.
Each station requires $ \frac{\SI{1000}{\bit}}{\SI{100}{\second}} = \SI{10}{\bit\per\second} $. Therefore, the maximum number of stations for this channel is $ \frac{\SI{10.3}{\kilo\bit\per\second}}{\SI{10}{\bit\per\second}} = 1030 $ stations.

\item The number of transmissions, $ E $, is $ E = e^G $. There are $ E - 1 $ intervals between $ E $ transmissions. Each interval is, as given, 4 slots, so the delay is then $ d = 4(e^G - 1) $. Thus, the delay versus throughput curve can be plotted:

insert plot here

\item Let $ i $ represent the number of attempts made by the stations to contend for the channel. On the first round ($ i = 1 $), there is only one backoff period (0), and there is a $ 2^{-(1-1)} = 1.0 $ chance of collision for that round. For the second round ($ i = 2 $), the backoff slots are uniformly distributed between 0 and $ 2^{2-1} - 1 = 1 $, and there is a $ 2^{-(2-1)} = 0.50 $ chance of collision for that round. For the $ (k - 1) $th round ($ i = k-1 $), the backoff slots are uniformly distributed between 0 and $ (2^{(k-1)-1} - 1) $, and there is a $ 2^{-[(k-1)-1]} $ chance of collision for that round. So, the probability of $ k-1 $ collisions up to the $ k $th contention round is given by the product of the chance of collision for each round up to the ($ k -1 $)th round:

$$ \prod_{i=1}^{k-1}2^{-(i-1)} $$

The probability of no collisions on the $ k $th round ($ i = k $) is given by 100\% minus the probability of collision for that round:

$$ (1-2^{-(k-1)}) $$

So the expression for the probability that contention ends on round $ k $ is given by:

$$ P_k = (1-2^{-(k-1)}) \times \prod_{i=1}^{k-1}2^{-(i-1)} $$

Simplified, knowing that $ \prod_{i=1}^{N}2^{-i} = 2^{\frac{-N \times (1 + N)}{2}}$:

$$ P_k = (1-2^{-(k-1)}) \times 2^{-\frac{(k-1)(k-2)}{2}} $$

Finally, the mean number of rounds per contention period is given by the expected value for the discrete random variable $ k $:

$$ E[k] = \sum_{k}k \times P_k $$

\item
\begin{enumerate}

\item When A is transmitting to B, no other communication can take place, because all other stations will have their transmissions interfered with by A's packet.

\item When B is transmitting to A, no other communications are possible. Although station D will not see B's packet, the other stations that could transmit to/from D (A, C, and E) will interfere with B's packet, as A would also receive packets from C and E.

\item When B is transmitting to C, E can transmit to D without interference. Station D cannot see B's packet, and C could not see E's packet. B would not see the packet from E, as its receiver is off when transmitting to C. E will shut off its receiver to transmit to D, so it will not have interference from B. Station A can see both packets, but because there is no communication to A, the interference does not matter.

\end{enumerate}

\item In the distributed coordination function in IEEE 802.11, a sender station contends for the channel by sending a Request-To-Send frame to the receiver. The sender will capture the channel if and only if it receives a Clear-To-Send frame from the receiver. The sender will not know if it succeeded until after the delays due to DIFS, the time to send RTS, SIFS, and the time to send CTS. The sender knows it is not clear to send if the CTS frame doesn't arrive after this detection time.

This is analogous to how detection time in CSMA-CD is given by $ 2t_{prop} $. Continuing with this analysis, assume that like in CSMA-CD, each station has a probability $ p $ to send. The probability of successful transmission is maximized when $ p = \frac{1}{n} $, giving a probability of success $ P_{max} = \frac{1}{e}$, and the mean number of contention slots in a contention period is $ e $, like the mean number of minislots to successfully send.

The maximum throughput occurs when all of the channel time is spent in transmission period (given by $ T_{data} + 2SIFS + T_{ack} $) followed by contention intervals (given by $ (DIFS + T_{rts} + SIFS + T_{cts}) $ times an average of $ e $ intervals). Thus, the capacity is given by:

$$ p_{max} = \frac{T_{data}}{T_{data} + 2SIFS + T_{ack} + e(DIFS + T_{rts} + SIFS + T_{cts})} $$

include pretty timeline plot here

\item In the case of 80\% of traffic being generated is for stations in the LAN and 20\% for stations on the other LAN, an Ethernet switch would be preferable. The Ethernet switch only forwards a frame to the other LAN if the destination is not in the first LAN, whereas the Ethernet hub would broadcast the frame to every station on both LANs. The hub would not be very efficient if 80\% of traffic on a LAN is meant to stay local, as the hub would waste bandwidth on the other LAN by broadcasting traffic not meant for it.

On the other hand, if 80\% of the traffic was meant to go to the other LAN, then the hub would be more effective, especially if the traffic is broadcast traffic; for example, an ARP request.

\end{enumerate}
\end{document}
