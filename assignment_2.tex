\documentclass{article}

\usepackage[scientific-notation=true, binary-units=true]{siunitx}
\usepackage{amsmath}
\sisetup{per-mode=fraction}%
\sisetup{scientific-notation=false}%

\title{SYSC 4602 Assignment 2}
\date{September 30th, 2016}
\author{Jessica Morris \(100882290\)}

\begin{document}

\maketitle

\begin{enumerate}

\item OC-1 speed is $ \SI{51.84}{\mega\bit\per\second} $, where one bit lasts $ \SI{1.929e-8}{\second} $. The time for the clock to drift by 1 bit is given by:

$$ t_{drift} = \frac{\SI{1.929e-8}{\second}}{\SI{1}{\nano\second\per\second}}$$
$$ t_{drift} = \SI{19.29}{\second} $$

Therefore, it takes a SONET clock 19.29 seconds to drift by 1 bit at OC-1 speed. This suggests that SONET clocks should be resynchronized every $ \SI{19.29}{\second} $, but resynchronizing at half of that time should be the maximum.

\item Consider that the delay for a circuit-switched network is due to the setup time, $s$, the propagation delay, $kd$, and and the transmission time, $\frac{x}{b}$. Determining the total delay for the circuit-switched network:

$$ d_{total} = s + kd + \frac{x}{b} $$

The delay for a packet-switched network, on the other hand, is determined by the propagation delay, $kd$, and the transmission time per hop. The transmission time per hop is given by the time for the first packet to fully transmit from source to destination, $x$ bits over $b$ bits per second, plus the time taken for $p$ packets to complete $k-1$ hops over $b$ bits per second. So, the total delay for the packet-switched network is given by:

$$ d_{total} = kd + \frac{x}{b} + \frac{(k-1)p}{b} $$

Since the two delays have $kd$ and $\frac{x}{b}$ in common, the packet-switched network will have a smaller delay than the circuit-switched network when $s < \frac{(k-1)p}{b}$.

\item To determine which stations transmitted, and the bits they sent, we multiply each element in the station's chip sequence by the received chips, and divide by 8:

$$ b_A = \frac{S \cdot A}{8} = \frac{(+1 -1 +3 +1 -1 +3 +1 +1)}{8} = 1 $$
$$ b_B = \frac{S \cdot B}{8} = \frac{(+1 -1 -3 -1 -1 -3 +1 -1)}{8} = -1 $$
$$ b_C = \frac{S \cdot C}{8} = \frac{(+1 +1 +3 +1 -1 -3 -1 -1)}{8} = 0 $$
$$ b_D = \frac{S \cdot D}{8} = \frac{(+1 +1 +3 -1 +1 +3 +1 -1)}{8} = 1 $$

So Station A transmitted a 1, Station B transmitted a -1, Station C did not transmit, and Station D transmitted a 1.

\item
\begin{enumerate}

\item For $k=2$, the second stage of connections is the bottleneck, and blocking could occur at the first stage. Thus, for $k=2$, 8 simultaneous connections are supported.

For $k=4$, since $k \geq n$, it is possible to reach a maximum of 16 simultaneous connections as long as rearrangement is allowed.

For $k=10$, by the Clos theorem, since $k \geq 2n-1$, a new connection can always be added without rearrangement. Since there are only 16 inputs and 16 outputs for the ten $4 \times 4$ second-stage switches, the maximum of 16 simultaneous connections is allowed under all circumstances.

\item Yes. Each input-output pair can be connected through any one of the $k$ second-stage switches, giving $k$ different ways to arrange the connections.

\end{enumerate}

\item
\begin{enumerate}

\item Using byte count:  00000101 01000111 11100011 11100000 01111110 (The first byte is the byte count for the frame, which is 5 including the byte for byte count.)

\item Using byte stuffing: 01000111 11100011 11100000 11100000 11100000 01111110
(This is A B ESC ESC ESC FLAG.) 

\item Using bit stuffing: 01 0001 1111 0100 0110 1111 1010 1110 0000

\end{enumerate}

\item Given $ G(x) = x^3 + 1 $, or G = 1001, and $ D = 1001101 $, the remainder calculation is as follows:
% Insert calc from my paper notes here

% Something something show an inverted bit gets picked up - still on paper

\item The checksum for 1001 1100 1010 0011 is given by:

$$ b_0 = 9 $$
$$ b_1 = 12 $$
$$ b_2 = 10 $$
$$ b_3 = 3 $$
$$ b_0 + b_1 + b_2 + b_3 = 36 = 6 \mod 15 $$
$$ 6 = 0110_2 $$
$$ -6 = 1010_2 $$
$$ checksum = 1010 $$

\item

\end{enumerate}
\end{document}
