\documentclass{article}

\usepackage[scientific-notation=true, binary-units=true]{siunitx}
\usepackage{amsmath}
\usepackage{tree-dvips}
\usepackage{bytefield}
\sisetup{per-mode=fraction}
\sisetup{scientific-notation=false}

\title{SYSC 4602 Assignment 4}
\date{December 9th, 2016}
\author{Jessica Morris \(100882290\)}

\begin{document}

\maketitle

\begin{enumerate}

\item Tabulating the distance vectors from B, D, and E, while taking into account C's distance from each of them gives the distance table:

\begin{center}
\begin{tabular}{ c|c|c|c }
    & B & D & E \\
    \hline
    A & 11 & 19 & 12 \\ \hline
    B & 6 & 15 & 11 \\ \hline
    D & 18 & 3 & 14 \\ \hline
    E & 12 & 12 & 5 \\ \hline
    F & 8 & 13 & 9 \\ \hline
\end{tabular}
\end{center}

Picking the minimum values from each row gives the new vector table:
\begin{center}
\begin{tabular}{ c|c }
    & Distance, Next Hop \\
    \hline
    A & 11, B \\ \hline
    B & 6, B \\ \hline
    D & 3, D \\ \hline
    E & 5, E \\ \hline
    F & 8, B \\ \hline
\end{tabular}
\end{center}

\item
\begin{enumerate}

\item Running the Dijkstra algorithm from Node 4 gives the following results:
\begin{center}
\begin{tabular}{ |c|c|c|c|c|c|c| }
    Step & Visited Set &  \multicolumn{5}{c|}{Node Distance} \\
     & & 1 & 2 & 3 & 5 & 6 \\
    \hline
    0 & \{4\} & 5 & 1 & 2 & 3 & $\infty$ \\ \hline
    1 & \{4, 1\} & 5 & 1 & 2 & 3 & $\infty$ \\ \hline
    2 & \{4, 1, 2\} & 4 & 1 & 2 & 3 & $\infty$ \\ \hline
    3 & \{4, 1, 2, 3\} & 4 & 1 & 2 & 3 & 3 \\ \hline
    4 & \{4, 1, 2, 3, 5\} & 4 & 1 & 2 & 3 & 3 \\ \hline
    5& \{4, 1, 2, 3, 5, 6\} & 4 & 1 & 2 & 3 & 3 \\ \hline
\end{tabular}
\end{center}

The shortest paths from Node 4 to other nodes are:
\begin{center}
\begin{tabular}{ c|c }
    Destination & Path\\
    \hline
    1 & $ 4 \rightarrow 2 \rightarrow 1 $ \\ \hline
    2 & $ 4 \rightarrow 2 $ \\ \hline
    3 & $ 4 \rightarrow 3 $ \\ \hline
    5 & $ 4 \rightarrow 5 $ \\ \hline
    6 & $ 4 \rightarrow 3 \rightarrow 6 $ \\ \hline
\end{tabular}
\end{center}

The spanning tree for Node 4 is:
\begin{center}
\begin{tabular}{ccc}
\node{e}{1} & \node{c}{3} & \node{f}{6} \\[3ex]
            & \node{a}{4} &             \\[3ex]
\node{b}{2} &             & \node{d}{5}
\end{tabular}
\nodeconnect{a}{b}
\nodeconnect{a}{d}
\nodecurve[t]{a}[b]{c}{10pt}
\nodecurve[t]{b}[b]{e}{10pt}
\nodecurve[r]{c}[l]{f}{10pt}
\end{center}

\item The routing table for node 4 is:

\begin{center}
\begin{tabular}{ c | c }
    Destination & Distance, Next Hop \\
    \hline
    1 & 4, 2 \\ \hline
    2 & 1, 2 \\ \hline
    3 & 3, 2 \\ \hline
    5 & 3, 5 \\ \hline
    6 & 3, 3 \\ \hline
\end{tabular}
\end{center}

\end{enumerate}

\item
\begin{enumerate}
\item The IP addresses for the six computers in the EE department, from right to left, are 111.111.1.1, 111.111.1.2, 111.111.1.3, 111.111.1.4, 111.111.1.5, 111.111.1.6. The subnet mask is 111.111.1.0/24.
The IP addresses for the six computers in the CS department, from right to left, are 111.111.2.1, 111.111.2.2, 111.111.2.3, 111.111.2.4, 111.111.2.5, 111.111.2.6. The subnet mask is 111.111.2.0/24.
The IP addresses of the router interface are [FIX ME]
\item The steps taken to transfer an IP datagram from an EE host on the left switch to an EE host on the right switch are:
\begin{itemize}
\item FIX ME
\end{itemize}
\item The steps taken to transfer an IP datagram from an EE host on the left switch to a CS host on the right switch are:
\begin{itemize}
\item FIX ME
\end{itemize}
\end{enumerate}

\item It is sufficient to add one new table entry: 29.18.0.0/22 for the new block. If an incoming packet's destination matches both 29.18.0.0/17 and 29.18.0.0/22, then the more specific rule (/22) will apply.

\item Designing the variable-length addressing scheme for this company requires working from the least-specific tier (AS Number) towards the most-specific tier (Department). Starting with the AS Number, we assign subnets from largest AS to smallest AS:
\begin{itemize}
\item AS3 has 3700 hosts, and the smallest size block that would fit this is 4096. This requires 12 bits to specify, so AS3 is mapped to 129.99.0.0/20, and its prefix is bits 16-20 are "0000".
\item AS1 has 2270 hosts, and the smallest size block that would fit this is also 4096. AS1 is mapped to 129.99.16.0/20, and its prefix is bits 16-20 are "0001".
\item AS2 has 185 hosts, and the smallest size block that would fit this is 256. AS2 is mapped to 129.99.32.0/24, and its prefix is bits 16-24 are "0010 0000".
\end{itemize}

A similar process is followed to assign the locations. For example, for AS1:
\begin{itemize}
\item Assigning Chicago Campus Building 2 first, as it has more hosts (1780 hosts). This will need a block size of 2048. Its subnet is then 129.99.16.0/21, and its prefix is bit 21 is set to "0".
\item Chicago Campus Building 1 has 490 hosts, so it is allocated a 512-address block. Its subnet is 129.99.24.0/23, and its prefix is bits 21-23 are set to "100".
\end{itemize}

The final addressing scheme is as follows:

\begin{tabular}{ c|c|c|c|c|c|c }
    AS Number & Subnet & Location & Subnet & Department & Subnet & Block Size \\ \hline
    % Gonna make this pretty at home
\end{tabular}

%http://www.vlsm-calc.net/

\item In a way, yes, it is possible with Network Address Translation. As long as the two (or more) routers can exchange their NAT translation table information every time a new entry is added to the table. CISCO has achieved this with their HSRP protocol.
\\ (Reference: http://www.cisco.com/c/en/us/support/docs/ip/hot-standby-router-protocol-hsrp/9234-hsrpguidetoc.html)

\item No. A pair of ports sets uniquely up a single connection, so $ (1, p) \longleftrightarrow (2, q) $ is the only possible connection between the two hosts/ports. Additional TCP connections would require additional port pairs.

\end{enumerate}
\end{document}
